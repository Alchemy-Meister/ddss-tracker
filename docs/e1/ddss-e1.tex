\documentclass[twoside,a4paper,10pt]{article}

\usepackage[english]{babel}
\usepackage[utf8]{inputenc}
\usepackage{amsmath}
\usepackage{graphicx}
%\usepackage[colorinlistoftodos]{todonotes}
%\usepackage{url}

\usepackage[hidelinks]{hyperref}
\usepackage{tabularx}
%\usepackage{ref}

%%%

\pagenumbering{arabic}
\usepackage{fancyhdr}

\pagestyle{fancy}
% Shows section number and name
\renewcommand{\sectionmark}[1]{\markright{#1}{}}
% Clear previous styles
\fancyhf{}
\fancyhead{}
\fancyhead[RO]{\thepage}
\fancyhead[LO]{\rightmark}
\fancyhead[RE]{BitTorrent Tracker: deliverable 1}
\fancyhead[LE]{\thepage}
\fancyfoot{}
% Other modifiers
%\fancyfoot[LE,RO]{\thepage}
%\fancyfoot[LO,CE]{Something}
%\fancyfoot[CO,RE]{Author Name}

\title{BitTorrent Tracker: deliverable 1}

\author{Irene Díez \and Jesus Sesma}

\begin{document}
\date{}
\maketitle

%\begin{abstract}
%Your abstract.
%\end{abstract}

\section{Architectural design}

Figure~\ref{fig:arch} shows the architectural design of the BitTorrent Tracker.

\begin{figure}[h]
  \centering
  \includegraphics[width=0.3\textwidth]{frog.jpg}
  \caption{\label{fig:arch}Bla blá}
\end{figure}

\section{Functionality}

The tracker can be divided into the following components:
% TODO: explain
\begin{itemize}
\item Master tracker
  \begin{itemize}
  \item Cluster Fault Tolerance System
  \item Peer Info Sender
  \item Master DB Fault Tolerance System
    \begin{itemize}
    \item DB Manager
    \item Master Consensus System
    \end{itemize}
  \end{itemize}
\item Tracker slaves
  \begin{itemize}
  \item Cluster Fault Tolerance System
  \item Master Election System
  \item Slave DB Fault Tolerance System
    \begin{itemize}
    \item DB Manager
    \item Slave Consensus System
    \end{itemize}
  \end{itemize}
\item Peers
  \begin{itemize}
  \item Peer Info Requester
  \item Download Manager
  \item DB Manager
  \end{itemize}
\end{itemize}

\begin{table}[h]
  \centering
  \begin{tabularx}{\linewidth}{X X l}
    Identifier & Entity & Description \\ \hline\hline
    Cluster Fault Tolerance System & Master tracker &  \\
    Master Consensus System & Tracker master &  \\
    Peer Info Sender & Master tracker &  \\
    Master DB Fault Tolerance System & Master tracker &  \\
    Slave DB Fault Tolerance System & Tracker slaves &  \\
    DB Manager & Master tracker, tracker slaves and peers &  \\
    Slave Consensus System & Tracker slaves &  \\
    Master Election System & Tracker slaves &  \\
    Peer Info Requester & Peers &  \\
    Download Manager & Peers &  \\
  \end{tabularx}
  \caption{\label{tab:fun-entities}Summary of the functionality
    implemented in each entity}
\end{table}
\section{Data schema}

Our system contemplates two different database schemas, on the one hand the
tracker's master and slaves will implement the schema shown in
figure~\ref{fig:schema-MS}. This schema has two tables: (i) \texttt{PEER-INFO}
where information regarding the peers' host and ports is stored; and (ii)
\texttt{CONTENTS}, where the tracker will store which peers have available some
specific content. Both tables' fields are self-descriptive.

\begin{figure}[h]
  
  \texttt{PEER-INFO (\underline{id:INTEGER}, host:VARCHAR(255), port:INTEGER)}
  
  \texttt{CONTENTS (\underline{sha1:STRING(40)}, \underline{peer\_id:INTEGER})}
  
  \centering
  \caption{\label{fig:schema-MS}DB schema for the tracker's master and slaves.}
\end{figure}

On the other hand, the peers will need to remember the
progress they have made during a download; thereby, they will store which chucks
they have downloaded so far for a specific file. It must be underlined that our
system will always transfer chunks of the same size, with the exception of the
last chunk, or when the content's size is inferior to our default chunk size.

This characteristic is implemented using the schema shown in
figure~\ref{fig:schema-P}, with the table \texttt{CHUNK}; its
fields are self-descriptive.

\begin{figure}[h]
  
  \texttt{PEER-INFO (\underline{sha1:STRING(40)}, offset:INTEGER)}
  
  \centering
  \caption{\label{fig:schema-P}DB schema for the tracker's peers.}
\end{figure}

\subsection{DB technology}

We have decided to use SQLite~\cite{sqlite} as our storage technology,
mainly because we have previous experience with it; but more importantly,
because this is a didactic project without the high availability and performance
needs that a professional project's database requires.

Regarding the use of a classic SQL database over the NoSQL paradigm, since
this project implies the synchronisation of the DB over the trackers, we think
that the data should be as normalised as possible and complying to the third
normal form (3NF); thereby, we discard NoSQL databases.

\section{Interaction model design}

\section{Failure model desing}

\section{Graphical interface}


\bibliographystyle{unsrt}
\bibliography{bib}

\end{document}


