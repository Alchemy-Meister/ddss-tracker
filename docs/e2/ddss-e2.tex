\documentclass[twoside,a4paper,10pt]{article}
\usepackage[top=2.54cm,bottom=2.54cm,left=2.54cm,right=2.54cm]{geometry}

\usepackage[english]{babel}
\usepackage[utf8]{inputenc}
\usepackage{amsmath}
\usepackage{graphicx}
%\usepackage[colorinlistoftodos]{todonotes}
\usepackage{url}

\usepackage[hidelinks]{hyperref}
\usepackage{tabularx}
\usepackage{placeins}
%\usepackage{ref}

%%%

\pagenumbering{arabic}
\usepackage{fancyhdr}

\pagestyle{fancy}
% Shows section number and name
\renewcommand{\sectionmark}[1]{\markright{#1}{}}
% Clear previous styles
\fancyhf{}
\fancyhead{}
\fancyhead[RO]{\thepage}
\fancyhead[LO]{\rightmark}
\fancyhead[RE]{BitTorrent Tracker: deliverable 2}
\fancyhead[LE]{\thepage}
\fancyfoot{}
% Other modifiers
%\fancyfoot[LE,RO]{\thepage}
%\fancyfoot[LO,CE]{Something}
%\fancyfoot[CO,RE]{Author Name}

\title{BitTorrent Tracker: deliverable 2\\
  Group 01}
\author{Irene Díez \and Jesus Sesma}

\begin{document}
\date{}
\maketitle

%\begin{abstract}
%Your abstract.
%\end{abstract}
\section{Package hierarchy}

Our project has the following package hierarchy:

\begin{itemize}
\item \texttt{bitTorrent}
  \begin{itemize}
  \end{itemize}
\item \texttt{general}
  \begin{itemize}
  \item \texttt{components}
  \end{itemize}
\item \texttt{peer}: this package contains the classes needed by the peers. 
  Since this deliverable contemplates the design of the tracker GUI and
  project skeleton it has been left empty.
\item \texttt{tracker}: this package contains the classes needed by the tracker.
  \begin{itemize}
  \item \texttt{db}: classes related to the DB management.
    \begin{itemize}
    \item \texttt{model}: classes part of the business model.
      \begin{itemize}
      \item \texttt{TrackerMember.java}: represents a member of the tracker.
      \end{itemize}
    \item \texttt{DBManager.java}: responsible for the insert, create and
      update operations.
    \end{itemize}
  \item \texttt{gui}: holds the classes related to the graphical user interface
    of the tracker.
    \begin{itemize}
    \item \texttt{BasicInfoPanel.java}: displays general information about
      the tracker, such as the current ID of the instance, IP and port etc.
    \item \texttt{ObserverJPanel.java}: abstract class that inherits from
      \texttt{JPanel} and implements the \texttt{Observer} interface.
    \item \texttt{PeerPanel.java}: 
    \item \texttt{TrackerPanel.java}:  
    \item \texttt{TrackerGUI.java}:  
    \end{itemize}
  \item \texttt{observers}
    \begin{itemize}
    \item \texttt{DBFaultToleranceObserver.java}
    \item \texttt{FaultToleranceObserver.java}
    \item \texttt{MasterElectionObserver.java}
    \item \texttt{TrackerObserver.java}
    \end{itemize}
  \item \texttt{subsys}: package that holds the subsystems of the tracker.
    
    \begin{itemize}
    \item \texttt{TrackerSubsystem.java}: abstract class that represents a
      subsystem of the tracker
    \item \texttt{cfts}
      \begin{itemize}
      \item \texttt{FaultToleranceSys.java}
      \item \texttt{IpIdTable.java}
      \end{itemize}
    \item \texttt{db}
      \begin{itemize}
      \item \texttt{DBFaultToleranceSys.java}
      \end{itemize}
    \item \texttt{election}
      \begin{itemize}
      \item \texttt{MasterElectionSys.java}
      \end{itemize}
    \end{itemize}
  \item \texttt{Launcher.java}: program entry point. 
  \end{itemize}
\end{itemize}

\section{Responsive design}

The tracker's gui follows the guidelines given by the responsive web design
paradigm, thereby the application's main window size and its components' size
can be adapted to a variety of screen sizes.

\section{Form validation}

\section{Exception management}

\section{Interaction}

%\bibliographystyle{unsrt}
%\bibliography{bib}

\end{document}


